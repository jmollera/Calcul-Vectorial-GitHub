\documentclass[catalan,a4paper,twoside,11pt]{article}

\pdfminorversion=7 % Crea un fitxer PDF v. 1.7
\pdfsuppresswarningpagegroup=1 % Suprimeix els warnings PDF multiple page...

\usepackage[T1]{fontenc}
\usepackage[utf8]{inputenc}
\DeclareUnicodeCharacter{013F}{\L.}  % ela geminada majúscula
\DeclareUnicodeCharacter{0140}{\l.}  % ela geminada minúscula

\usepackage{newpxtext}
\usepackage{eulerpx}

\usepackage{listings} % Llistats de programes
\renewcommand{\lstlistingname}{Llistat}
\lstdefinelanguage{HPPRIME}
{   morekeywords={EXPORT,BEGIN,END,LOCAL,RETURN,IF,THEN,ELSE,FOR,FROM,TO,DO},
	morekeywords=[2]{rectangular_coordinates, polar_coordinates},
	sensitive=true,
	morecomment=[l]{//},
	morestring=[b]"
}
\lstset{
	basicstyle=\small\ttfamily,
	xleftmargin=4mm,
	xrightmargin=0mm,
	belowskip=5pt,
	columns=fixed,
	breaklines=true,
	keywordstyle=\bfseries\textcolor[rgb]{0.0,0.0,0.75},
	keywordstyle=[2]\textcolor[rgb]{0.5,0.0,0.0},
	stringstyle=\ttfamily\textcolor[rgb]{0.0,0.5,0.0},
	commentstyle=\textcolor[rgb]{0.55,0.31,0.04},
	showspaces=false,
	showstringspaces=false,	
	numberstyle=\tiny,
	numbersep=4pt,
	framerule=1pt,
	numbers=left,
	frame=lines,
	language=HPPRIME
}

\usepackage{beccari-v7}
\allowdisplaybreaks[4] % Permet dividir múltiples equacions entre dues pàgines

\usepackage{babel}
\usepackage{varioref}

\usepackage[final]{graphicx}
\graphicspath{{Imatges/}}

\usepackage{booktabs}
\usepackage{multicol}

\usepackage{transparent} % Necessari pels fitxers TeX generats per Inkscape

\usepackage{caption} % Capçaleres de taules i peus de figures
\captionsetup{hypcap=false,labelsep=quad,skip=10pt,format=hang}
\captionsetup[table]{position=above}
\captionsetup[figure]{position=below}

\setlength{\parskip}{1.5ex plus0.5ex minus0.5ex}     % separació entre paràgrafs
\setlength{\parindent}{0mm}

% format de les pàgines
\setlength{\hoffset}{0mm} \setlength{\oddsidemargin}{0mm}
\setlength{\evensidemargin}{-10.8mm} \setlength{\textwidth}{170mm}
\setlength{\marginparsep}{2mm} \setlength{\marginparwidth}{12mm}
\setlength{\voffset}{0mm} \setlength{\textheight}{230mm}
\setlength{\topmargin}{0mm}

\allowdisplaybreaks[1]



\usepackage[pdftex,
bookmarks=true,bookmarksopenlevel=0,bookmarksopen=true,bookmarksnumbered=true,
pdftitle={Càlcul Vectorial},pdfauthor={Josep
Mollera},pdfview={FitH},pdfstartview={FitH}, pdfpagelabels=true,
colorlinks=true,linkcolor=black,plainpages=false]{hyperref}


\begin{document}

\title{Càlcul Vectorial en $\varmathbb{R}^3$}
\author{Josep Mollera Barriga}
\date{Versió 13 (\today)}
\maketitle


\section{Introducció}

Aquest escrit és una breu introducció al càlcul vectorial en $\varmathbb{R}^3$, és a dir, en tres dimensions. Es presenten de forma directa ---sense demostracions--- operadors, relacions i teoremes vectorials.

S'utilitzen els tres tipus de coordenades més usuals:  cartesianes, 
cilíndriques i esfèriques.

Totes les versions d'aquest text, així com els programes de la calculadora \textsf{HP Prime} inclosos, poden trobar-se a GitHub: \href{https://github.com/jmollera/Calcul-Vectorial-GitHub}{https://github.com/jmollera/Calcul-Vectorial-GitHub}.


\section{Definicions}

Es dona, en primer lloc, la definició de les variables i funcions que s'utilitzen en aquest document.

\begin{list}{}
{\setlength{\labelwidth}{16mm}
\setlength{\leftmargin}{20mm}\setlength{\labelsep}{4mm}}
   \item[$V$] Volum d'integració.

   \item[$S$] Superfície d'integració.

   \item[$C$] Corba d'integració.

   \item[$\diff\tau$] Diferencial de volum del volum V.

   \item[$\boldsymbol{\diff a}$] Vector diferencial de superfície de la superfície $S$.

   $\boldsymbol{\diff a}$ és perpendicular a $S$.

   \item[$\boldsymbol{\diff l}$] Vector diferencial de longitud de la corba
   $C$.

   $\boldsymbol{\diff l}$ és tangent a $C$.

   \item[$x,y,z$] Coordenades d'un punt en un sistema de coordenades cartesianes.

   \item[$\rho,\varphi,z$] Coordenades d'un punt en un sistema de   coordenades cilíndriques.

   \item[$r,\theta,\varphi$] Coordenades d'un punt en un sistema de   coordenades esfèriques.

   \item[$u,v,w$] Coordenades d'un punt en un sistema de   coordenades qualssevol.
   \item[$\boldsymbol{e_x},\boldsymbol{e_y},\boldsymbol{e_z}$]
   Vectors directors d'un sistema de  coordenades    cartesianes.

   \item[$\boldsymbol{e_\rho},\boldsymbol{e_\varphi},\boldsymbol{e_z}$] Vectors directors d'un sistema de   coordenades cilíndriques.

   \item[$\boldsymbol{e_r},\boldsymbol{e_\theta},\boldsymbol{e_\varphi}$] Vectors directors d'un sistema de   coordenades esfèriques.

   \item[$\boldsymbol{e_u},\boldsymbol{e_v},\boldsymbol{e_w}$]
   Vectors directors d'un sistema de  coordenades   qualssevol.

   \item[$P, Q$] Punts en $\varmathbb{R}^3$.

   \item[$\boldsymbol{A,B,C}$] Vectors en $\varmathbb{R}^3$.

   \item[$\alpha$] Angle entre dos vectors en $\varmathbb{R}^3$.

   \item[$f,g$] Funcions escalars.

   $f,g \colon \varmathbb{R}^3\rightarrow\varmathbb{R}$

   \item[$\boldsymbol{F,G}$] Funcions vectorials.

   $\boldsymbol{F,G}\colon\varmathbb{R}^3\rightarrow\varmathbb{R}^3$

   \item[$\boldsymbol{\nabla}f$] Gradient de la funció escalar $f$.

   $\boldsymbol{\nabla}f\colon\varmathbb{R}\rightarrow\varmathbb{R}^3$

   \item[$\boldsymbol{\nabla\cdot F}$] Divergència de la funció vectorial $\boldsymbol{F}$.

   $\boldsymbol{\nabla\cdot F}\colon \varmathbb{R}^3\rightarrow\varmathbb{R}$

   \item[$\boldsymbol{\nabla\times F}$] Rotacional de la funció vectorial $\boldsymbol{F}$.

   $\boldsymbol{\nabla\times F}\colon   \varmathbb{R}^3\rightarrow\varmathbb{R}^3$

   \item[$\boldsymbol{\nabla^2}f$] Laplacià de la funció escalar $f$.

   $\boldsymbol{\nabla^2}f\colon \varmathbb{R}\rightarrow\varmathbb{R}$

   \item[$\boldsymbol{\nabla^2F}$] Laplacià de la funció vectorial $\boldsymbol{F}$.

    $\boldsymbol{\nabla^2F}\colon \varmathbb{R}^3\rightarrow\varmathbb{R}^3$
\end{list}

\newcommand{\va}{\ensuremath{\,\boldsymbol{e_x}}}
\newcommand{\vb}{\ensuremath{\,\boldsymbol{e_y}}}
\newcommand{\vc}{\ensuremath{\,\boldsymbol{e_z}}}
\section{Sistemes de Coordenades}

Es representen en la Figura \vref{pic:coord-cart-cil-esf} tres dels
sistemes de coordenades més utilitzats: el cartesià, el
cilíndric i l'esfèric. L'orientació dels eixos i els orígens dels angles corresponen a un sistema de coordenades destre, tal com es defineix en la norma ISO 80000-2 \textit{Quantities and units --- Part 2: Mathematics}.

En el sistema de coordenades cartesianes els vectors directors
tenen una orientació fixa, mentre que en els sistemes de
coordenades cilíndriques i esfèriques, els vectors
directors tenen una orientació variable que depèn del punt
$P$ al qual ens estiguem referint.


\begin{center}
	\input{Imatges/Coordenades.pdf_tex}
	\captionof{figure}{Coordenades cartesianes, cilíndriques i esfèriques}
	\label{pic:coord-cart-cil-esf}
\end{center}


En la Taula \vref{taula:coord} es donen els rangs de les coordenades de cadascun d'aquests tres sistemes.

\begin{center}
	\captionof{table}{Rangs de les Coordenades}
	\label{taula:coord}
	\begin{tabular}{lllll}
   \toprule[1pt]
   \text{Cartesianes} &\quad\quad&   \text{Cilíndriques} &\quad\quad& \text{Esfèriques}
   \\
   \midrule
      $x\in(-\infty,\infty)$ &\quad\quad&   $\rho\in[0,\infty)$      & \quad\quad& $r\in[0,\infty)$  \\
     $y\in(-\infty,\infty)$ &\quad\quad&    $\varphi\in[0,2\pi)$     & \quad\quad& $\theta\in[0,\pi]$ \\
     $z\in(-\infty,\infty)$ &\quad\quad&    $z\in(-\infty,\infty)$  & \quad\quad& $\varphi\in[0,2\pi)$ \\
   \bottomrule[1pt]
   \end{tabular}
\end{center}

\subsection{Relacions entre les coordenades cartesianes i
cilíndriques}

Les coordenades cartesianes  d'un punt $P(x,y,z)$ s'obtenen a partir
de les seves coordenades cilíndriques $P(\rho,\varphi,z)$,
mitjançant les relacions següents:
\begin{subequations}\begin{align}
    x &=\rho\cos\varphi \\ y &=\rho\sin\varphi \\ z &=z
\end{align}\end{subequations}

Les coordenades  cilíndriques  d'un punt $P(\rho,\varphi,z)$
s'obtenen a partir de les seves coordenades cartesianes $P(x,y,z)$,
mitjançant les relacions següents:
\begin{subequations}\begin{align}
    \rho &= \sqrt{x^2+y^2}\\
    \varphi &=  \arctan\frac{y}{x}\\
    z &= z
\end{align}\end{subequations}

L'expressió en coordenades cartesianes dels vectors directors d'un sistema de coordenades  cilíndriques $\boldsymbol{e_\rho},\boldsymbol{e_\varphi},\boldsymbol{e_z}$, en un punt $P(\rho,\varphi,z)$ donat en coordenades cilíndriques, és:
\begin{subequations}\begin{align}
    \boldsymbol{e_\rho} &=\cos\varphi\va+\sin\varphi\vb \\
    \boldsymbol{e_\varphi} &=-\sin\varphi\va+\cos\varphi\vb \\
    \boldsymbol{e_z} &=\vc
\end{align}\end{subequations}

L'expressió en coordenades cilíndriques dels vectors directors d'un sistema de coordenades  cartesianes $\va,\vb,\vc$, en un punt $P(\rho,\varphi,z)$ donat en coordenades cilíndriques, és:
\begin{subequations}\begin{align}
    \va &=\cos\varphi\,\boldsymbol{e_\rho}-\sin\varphi\,\boldsymbol{e_\varphi} \\
    \vb &=\sin\varphi\,\boldsymbol{e_\rho}+\cos\varphi\,\boldsymbol{e_\varphi} \\
    \vc &=\boldsymbol{e_z}
\end{align}\end{subequations}

Les components d'un vector en coordenades cartesianes ($A_x, A_y, A_z)$ s'obtenen a partir de les seves components en coordenades cilíndriques $(A_\rho, A_\varphi, A_z)$, en un punt $P(\rho,\varphi,z)$ donat en coordenades cilíndriques, mitjançant les relacions següents:
\begin{subequations}\begin{align}
    A_x &=A_\rho \cos\varphi -A_\varphi\sin\varphi \\
    A_y &=A_\rho\sin\varphi +A_\varphi\cos\varphi\\
    A_z &= A_z
\end{align}\end{subequations}

Les components d'un vector en coordenades cilíndriques $(A_\rho, A_\varphi, A_z)$ s'obtenen a partir de les seves components en coordenades cartesianes ($A_x, A_y, A_z)$, en un punt $P(\rho,\varphi,z)$ donat en coordenades cilíndriques, mitjançant les relacions següents:
\begin{subequations}\begin{align}
    A_\rho &=  A_x\cos\varphi+A_y\sin\varphi\\
    A_\varphi &= -A_x\sin\varphi+A_y\cos\varphi \\
    A_z &= A_z
\end{align}\end{subequations}


\subsection{Relacions entre les coordenades cartesianes i
esfèriques}

Les coordenades cartesianes  d'un punt $P(x,y,z)$ s'obtenen a partir
de les seves coordenades esfèriques $P(r,\theta,\varphi)$,
mitjançant les relacions següents:
\begin{subequations}\begin{align}
    x &=r\sin\theta\cos\varphi \\ y &=r\sin\theta\sin\varphi \\ z &=r\cos\theta
\end{align}\end{subequations}

Les coordenades  esfèriques  d'un punt $P(r,\theta,\varphi)$
s'obtenen a partir de les seves coordenades cartesianes $P(x,y,z)$,
mitjançant les relacions següents:
\begin{subequations}\begin{align}
    r &=\sqrt{x^2+y^2+z^2}\\
    \theta&=\arctan{\frac{\sqrt{x^2+y^2}}{z}}=\arccos\frac{z}{\sqrt{x^2+y^2+z^2}}\\[1mm]
    \varphi &=\arctan\frac{y}{x}
\end{align}\end{subequations}


L'expressió en coordenades cartesianes dels vectors directors d'un sistema de coordenades  esfèriques $\boldsymbol{e_r},\boldsymbol{e_\theta},\boldsymbol{e_\varphi}$, en un punt $P(r,\theta,\varphi)$ donat en coordenades esfèriques, és:
\begin{subequations}\begin{align}
    \boldsymbol{e_r} &=\sin\theta\cos\varphi\va+ \sin\theta\sin\varphi\vb+\cos\theta\vc\\
    \boldsymbol{e_\theta} &=\cos\theta\cos\varphi\va+
    \cos\theta\sin\varphi\vb-\sin\theta\vc\\
    \boldsymbol{e_\varphi}&=-\sin\varphi\va+\cos\varphi\vb
\end{align}\end{subequations}

L'expressió en coordenades esfèriques dels vectors directors d'un sistema de coordenades  cartesianes $\va,\vb,\vc$, en un punt $P(r,\theta,\varphi)$ donat en coordenades esfèriques, és:
\begin{subequations}\begin{align}
    \va &=\sin\theta\cos\varphi\,\boldsymbol{e_r}+
    \cos\theta\cos\varphi\,\boldsymbol{e_\theta}-\sin\varphi\,\boldsymbol{e_\varphi}\\
    \vb &=\sin\theta\sin\varphi\,\boldsymbol{e_r}+
    \cos\theta\sin\varphi\,\boldsymbol{e_\theta}+\cos\varphi\,\boldsymbol{e_\varphi}\\
    \vc &=\cos\theta\,\boldsymbol{e_r}-\sin\theta\,\boldsymbol{e_\theta}
\end{align}\end{subequations}

Les components d'un vector en coordenades cartesianes $(A_x, A_y, A_z)$ s'obtenen a partir de les seves components en coordenades esfèriques $(A_r, A_\theta, A_\varphi)$, en un punt $P(r,\theta,\varphi)$ donat en coordenades esfèriques, mitjançant les relacions següents:
\begin{subequations}\begin{align}
    A_x &= A_r\sin\theta\cos\varphi+A_\theta\cos\theta\cos\varphi-A_\varphi\sin\varphi \\
    A_y &= A_r\sin\theta\sin\varphi+A_\theta\cos\theta\sin\varphi+A_\varphi\cos\varphi\\
    A_z &= A_r\cos\theta-A_\theta\sin\theta
\end{align}\end{subequations}

Les components d'un vector en coordenades esfèriques $(A_r, A_\theta, A_\varphi)$ s'obtenen a partir de les seves components en coordenades cartesianes $(A_x, A_y, A_z)$, en un punt $P(r,\theta,\varphi)$ donat en coordenades esfèriques, mitjançant les relacions següents:
\begin{subequations}\begin{align}
    A_r &=  A_x\sin\theta\cos\varphi+A_y\sin\theta\sin\varphi+A_z\cos\theta\\
    A_\theta &=  A_x\cos\theta\cos\varphi+A_y\cos\theta\sin\varphi-A_z\sin\theta\\
    A_\varphi &= -A_x\sin\varphi+A_y\cos\varphi
\end{align}\end{subequations}


\subsection{Relacions entre les coordenades cilíndriques i
esfèriques}

Les coordenades cilíndriques  d'un punt $P(\rho,\varphi,z)$
s'obtenen a partir de les seves coordenades esfèriques
$P(r,\theta,\varphi)$, mitjançant les relacions següents:
\begin{subequations}\begin{align}
    \rho &=r\sin\theta \\ \varphi &=\varphi \\z &=r\cos\theta
\end{align}\end{subequations}

Les coordenades  esfèriques  d'un punt $P(r,\theta,\varphi)$
s'obtenen a partir de les seves coordenades cilíndriques
$P(\rho,\varphi,z)$, mitjançant les relacions següents:
\begin{subequations}\begin{align}
    r &=\sqrt{\rho^2+z^2}\\
    \theta &=\arctan\frac{\rho}{z}\\
    \varphi &=\varphi
\end{align}\end{subequations}

L'expressió en coordenades cilíndriques dels vectors directors d'un sistema de coordenades  esfèriques $\boldsymbol{e_r},\boldsymbol{e_\theta},\boldsymbol{e_\varphi}$, en un punt $P(r,\theta,\varphi)$ donat en coordenades esfèriques, és:
\begin{subequations}\begin{align}
    \boldsymbol{e_r} &=\sin\theta\,\boldsymbol{e_\rho}+\cos\theta\,\boldsymbol{e_z}\\
    \boldsymbol{e_\theta}
    &=\cos\theta\,\boldsymbol{e_\rho}-\sin\theta\,\boldsymbol{e_z}\\
    \boldsymbol{e_\varphi}&=\boldsymbol{e_\varphi}
\end{align}\end{subequations}

L'expressió en coordenades esfèriques dels vectors directors d'un sistema de coordenades  cilíndriques $\boldsymbol{e_\rho},\boldsymbol{e_\varphi},\boldsymbol{e_\theta}$, en un punt $P(r,\theta,\varphi)$ donat en coordenades esfèriques, és:
\begin{subequations}\begin{align}
    \boldsymbol{e_\rho} &=\sin\theta\,\boldsymbol{e_r}+
    \cos\theta\,\boldsymbol{e_\theta}\\
    \boldsymbol{e_\varphi}&=\boldsymbol{e_\varphi}\\
    \boldsymbol{e_z} &=\cos\theta\,\boldsymbol{e_r}-
    \sin\theta\,\boldsymbol{e_\theta}
\end{align}\end{subequations}

Les components d'un vector en coordenades cilíndriques  $(A_\rho, A_\varphi, A_z)$ s'obtenen a partir de les seves components en coordenades esfèriques $(A_r, A_\theta, A_\varphi)$, en un punt $P(r,\theta,\varphi)$ donat en coordenades esfèriques, mitjançant les relacions següents:
\begin{subequations}\begin{align}
    A_\rho &= A_r\sin\theta+A_\theta\cos\theta \\
    A_\varphi &= A_\varphi\\
    A_z &= A_r\cos\theta-A_\theta\sin\theta
\end{align}\end{subequations}

Les components d'un vector en coordenades esfèriques $(A_r, A_\theta, A_\varphi)$ s'obtenen a partir de les seves components en coordenades cilíndriques $(A_\rho, A_\varphi, A_z)$, en un punt $P(r,\theta,\varphi)$ donat en coordenades esfèriques, mitjançant les relacions següents:
\begin{subequations}\begin{align}
    A_r &=  A_\rho\sin\theta+A_z\cos\theta\\
    A_\theta &=  A_\rho\cos\theta-A_z\sin\theta\\
    A_\varphi &= A_\varphi
\end{align}\end{subequations}


\section{Operacions  Bàsiques}
\renewcommand{\va}{\ensuremath{\,\boldsymbol{e_u}}}
\renewcommand{\vb}{\ensuremath{\,\boldsymbol{e_v}}}
\renewcommand{\vc}{\ensuremath{\,\boldsymbol{e_w}}}

A partir de les components $(A_u,A_v,A_w)$ i $(B_u,B_v,B_w)$
de dos vectors $\boldsymbol{A}$ i $\boldsymbol{B}$ en un sistema de coordenades qualssevol, tenim:

\subsection{Mòdul}
\vspace{-5mm}
\begin{equation}
    |\boldsymbol{A}|=  \sqrt{A_u^2 + A_v^2 + A_w^2}
\end{equation}

\subsection{Addició i subtracció}
\vspace{-5mm}
\begin{subequations}\begin{align}
    \boldsymbol{A+B}= (A_u+B_u)\va + (A_v+B_v)\vb + (A_w+B_w)\vc \\
    \boldsymbol{A-B}= (A_u-B_u)\va + (A_v-B_v)\vb + (A_w-B_w)\vc
\end{align}\end{subequations}

\subsection{Producte escalar}
\vspace{-5mm}
\begin{align}
    \boldsymbol{A\cdot B} &= A_u B_u + A_v B_v + A_w B_w\\
    \boldsymbol{A\cdot B} &=|\boldsymbol{A}| \, |\boldsymbol{B}| \cos\alpha\\
    \boldsymbol{A\cdot B} &=\boldsymbol{B\cdot A}\\
    \boldsymbol{A\cdot(B+C)} &= \boldsymbol{A\cdot B+ A\cdot C}
\end{align}

El producte escalar de dos vectors perpendiculars  és nul.

\subsection{Producte vectorial}
\vspace{-7mm}
\begin{align}
    \boldsymbol{A\times B} &= (A_v B_w - A_w B_v)\va + (A_w B_u - A_u B_w)\vb +
    (A_u B_v - A_v B_u)\vc \\
    |\boldsymbol{A\times B}| &=|\boldsymbol{A}| \, |\boldsymbol{B}| \sin\alpha\\
    \boldsymbol{A\times B} &=-(\boldsymbol{B\times A})\\
    \boldsymbol{A\times(B+C)} &= \boldsymbol{A\times B+ A\times C}
\end{align}

El producte vectorial de dos vectors paraŀlels  és nul.

El producte vectorial de dos vectors és un altre vector, el
sentit del qual és definit per la regla del caragol: és el sentit
d'avanç que té un caragol, amb el seu eix perpendicular al
pla format pels vectors  $\boldsymbol{A}$ i $\boldsymbol{B}$, quan
gira en el sentit que portaria el primer vector  $\boldsymbol{A}$ a
trobar el segon vector $\boldsymbol{B}$, utilitzant el menor angle
possible.

\subsection{Derivada temporal}
\begin{equation}
    \deriv{\boldsymbol{A}}{t} = \deriv{A_u}{t}\va +
    \deriv{A_v}{t}\vb + \deriv{A_w}{t}\vc
\end{equation}


\section{Càlcul Diferencial }

\subsection{Coordenades cartesianes}
\renewcommand{\va}{\ensuremath{\,\boldsymbol{e_x}}}
\renewcommand{\vb}{\ensuremath{\,\boldsymbol{e_y}}}
\renewcommand{\vc}{\ensuremath{\,\boldsymbol{e_z}}}

Es donen a continuació les derivades parcials dels vectors directors $\va$, $\vb$ i $\vc$, respecte $x$, $y$ i $z$:
\begin{subequations}
\begin{align}
   \pderiv{\va}{x} &= 0 & \pderiv{\va}{y} &= 0 & \pderiv{\va}{z} &= 0 \\
   \pderiv{\vb}{x} &= 0 & \pderiv{\vb}{y} &= 0 & \pderiv{\vb}{z} &= 0 \\
   \pderiv{\vc}{x} &= 0 & \pderiv{\vc}{y} &= 0 & \pderiv{\vc}{z} &= 0
\end{align}
\end{subequations}

En les equacions següents, $(F_x,F_y,F_z)$  són les
components de la funció  vectorial $\boldsymbol{F}$ en un sistema de
coordenades cartesianes. En aquestes coordenades tenim:

\begin{align}
    \boldsymbol{\diff l} &= \diff x \va + \diff y \vb + \diff z \vc\\[1ex]
    \boldsymbol{\diff a} &= \diff x \diff y \vc \qquad\text{(en un pla
    paraŀlel al X-Y)}\\[1ex]
    \boldsymbol{\diff a} &= \diff x \diff z \vb \qquad\text{ (en un pla
    paraŀlel al X-Z)}\\[1ex]
    \boldsymbol{\diff a} &= \diff y \diff z \va \qquad\text{ (en un pla
    paraŀlel al Y-Z)}\\[1ex]
    \diff\tau &= \diff x \diff y \diff z\\[1ex]
    \boldsymbol{\nabla}f &= \pderiv{f}{x}\va + \pderiv{f}{y}\vb
    + \pderiv{f}{z}\vc\\[1ex]
    \boldsymbol{\nabla\cdot F} &= \pderiv{F_x}{x} + \pderiv{F_y}{y}
    + \pderiv{F_z}{z}\\[1ex]
    \boldsymbol{\nabla\times F} &= \left[\pderiv{F_z}{y}-\pderiv{F_y}{z}\right]\va +
    \left[\pderiv{F_x}{z}-\pderiv{F_z}{x}\right]\vb +
    \left[\pderiv{F_y}{x}-\pderiv{F_x}{y}\right]\vc\\[1ex]
    \boldsymbol{\nabla^2}f &= \pderiv[2]{f}{x} + \pderiv[2]{f}{y}+ \pderiv[2]{f}{z}
\end{align}

\subsection{Coordenades cilíndriques}

\renewcommand{\va}{\ensuremath{\,\boldsymbol{e_\rho}}}
\renewcommand{\vb}{\ensuremath{\,\boldsymbol{e_\varphi}}}
\renewcommand{\vc}{\ensuremath{\,\boldsymbol{e_z}}}

Es donen a continuació les derivades parcials dels vectors directors $\va$, $\vb$ i $\vc$, respecte $\rho$, $\varphi$ i $z$:
\begin{subequations}
\begin{align}
   \pderiv{\va}{\rho} &= 0 & \pderiv{\va}{\varphi} &= \vb  & \pderiv{\va}{z} &= 0 \\
   \pderiv{\vb}{\rho} &= 0 & \pderiv{\vb}{\varphi} &= -\va & \pderiv{\vb}{z} &= 0 \\
   \pderiv{\vc}{\rho} &= 0 & \pderiv{\vc}{\varphi} &= 0    & \pderiv{\vc}{z} &= 0
\end{align}
\end{subequations}

En les equacions següents, $(F_\rho,F_\varphi,F_z)$  són les
components de la funció vectorial   $\boldsymbol{F}$ en un sistema de
coordenades cilíndriques. En aquestes coordenades tenim:
\begin{align}
    \boldsymbol{\diff l} &= \diff \rho \va + \rho\diff \varphi \vb + \diff z \vc\\[1ex]
    \boldsymbol{\diff a} &= \rho\diff \varphi \diff z \va\qquad\text{(en una superfície cilíndrica)}\\[1ex]
    \diff\tau &= \rho \diff \rho \diff \varphi \diff z\\[1ex]
    \boldsymbol{\nabla}f &= \pderiv{f}{\rho}\va + \frac{1}{\rho}\pderiv{f}{\varphi}\vb
    + \pderiv{f}{z}\vc\\[1ex]
    %\displaybreak
    \boldsymbol{\nabla\cdot F} &= \frac{1}{\rho}\pderiv{}{\rho}(\rho F_\rho) +
    \frac{1}{\rho}\pderiv{F_\varphi}{\varphi}+ \pderiv{F_z}{z}\\[1ex]
    \boldsymbol{\nabla\times F} &= \left[\frac{1}{\rho}\pderiv{F_z}{\varphi}-
    \pderiv{F_\varphi}{z}\right]\va +
    \left[\pderiv{F_\rho}{z}-\pderiv{F_z}{\rho}\right]\vb +
    \frac{1}{\rho}\left[\pderiv{}{\rho}(\rho F_\varphi)-\pderiv{F_\rho}{\varphi}\right]\vc\\[1ex]
    \boldsymbol{\nabla^2}f &= \frac{1}{\rho}\pderiv{}{\rho}\left(\rho\pderiv{f}{\rho}
    \right)
    + \frac{1}{\rho^2}\pderiv[2]{f}{\varphi}+ \pderiv[2]{f}{z}
\end{align}


\subsection{Coordenades esfèriques}

\renewcommand{\va}{\ensuremath{\,\boldsymbol{e_r}}}
\renewcommand{\vb}{\ensuremath{\,\boldsymbol{e_\theta}}}
\renewcommand{\vc}{\ensuremath{\,\boldsymbol{e_\varphi}}}

Es donen a continuació les derivades parcials dels vectors directors $\va$, $\vb$ i $\vc$, respecte  $r$, $\theta$ i $\varphi$:
\begin{subequations}
\begin{align}
   \pderiv{\va}{r} &= 0 & \pderiv{\va}{\theta} &= \vb  & \pderiv{\va}{\varphi} &= \sin\theta\vc \\
   \pderiv{\vb}{r} &= 0 & \pderiv{\vb}{\theta} &= -\va & \pderiv{\vb}{\varphi} &= \cos\theta\vc \\
   \pderiv{\vc}{r} &= 0 & \pderiv{\vc}{\theta} &= 0    & \pderiv{\vc}{\varphi} &= -\sin\theta\va-\cos\theta\vb
\end{align}
\end{subequations}

En les equacions següents, $(F_r,F_\theta,F_\varphi)$  són
les components de la funció  vectorial  $\boldsymbol{F}$ en un sistema de
coordenades esfèriques. En aquestes coordenades tenim:
\begin{align}
    \boldsymbol{\diff l} &= \diff r \va + r\diff\theta \vb + r \sin\theta\diff \varphi \vc\\[1ex]
    \boldsymbol{\diff a} &= r^2 \sin\theta \diff \theta \diff \varphi \va\qquad\text{(en una superfície esfèrica)}\\[1ex]
    \diff\tau &= r^2 \sin\theta\diff r \diff \theta \diff \varphi\\[1ex]
    \boldsymbol{\nabla}f &= \pderiv{f}{r}\va + \frac{1}{r}\pderiv{f}{\theta}\vb
    + \frac{1}{r\sin\theta}\pderiv{f}{\varphi}\vc\\[1ex]
    \boldsymbol{\nabla\cdot F} &= \frac{1}{r^2}\pderiv{}{r}(r^2 F_r) +
    \frac{1}{r\sin\theta}\pderiv{}{\theta}(F_\theta\sin\theta)+
    \frac{1}{r\sin\theta}\pderiv{F_\varphi}{\varphi}\\[1ex]
    \boldsymbol{\nabla\times F} &= \frac{1}{r\sin\theta}\left[\pderiv{}{\theta}
    (F_\varphi\sin\theta)-\pderiv{F_\theta}{\varphi}\right]\va +
    \frac{1}{r}\left[\frac{1}{\sin\theta}\pderiv{F_r}{\varphi}-
    \pderiv{}{r}(r F_\varphi)\right]\vb +\frac{1}{r}\left[\pderiv{}{r}(r F_\theta)-\pderiv{F_r}{\theta}\right]\vc\\[1ex]
    \boldsymbol{\nabla^2}f &= \frac{1}{r^2}\pderiv{}{r}\left(r^2\pderiv{f}{r}
    \right) + \frac{1}{r^2\sin\theta}\pderiv{}{\theta}\left(\sin\theta\pderiv{f}{\theta}\right) +
    \frac{1}{r^2\sin^2\theta}\pderiv[2]{f}{\varphi}
\end{align}


\section{Relacions}

\subsection{Operadors}
\renewcommand{\va}{\ensuremath{\,\boldsymbol{e_x}}}
\renewcommand{\vb}{\ensuremath{\,\boldsymbol{e_y}}}
\renewcommand{\vc}{\ensuremath{\,\boldsymbol{e_z}}}
Els operadors $\boldsymbol{\nabla}$ i $\boldsymbol{\nabla^2}$ es defineixen en coordenades cartesianes, segons les relacions següents:
\begin{align}
    \boldsymbol{\nabla} &\equiv \va \pderiv{}{x} + \vb \pderiv{}{y}
    + \vc \pderiv{}{z}\\[0.5ex]
    \boldsymbol{\nabla^2} &\equiv \pderiv[2]{}{x} + \pderiv[2]{}{y}+ \pderiv[2]{}{z}
\end{align}

\subsection{Identitats}
Es compleixen les identitats següents:
\begin{align}
    \boldsymbol{\nabla^2}f &=\boldsymbol{\nabla\cdot}(\boldsymbol{\nabla}f)\\
    \boldsymbol{\nabla^2 F}  &= \boldsymbol{\nabla}(\boldsymbol{\nabla\cdot F})
    - \boldsymbol{\nabla\times}(\boldsymbol{\nabla\times F})\\
    \boldsymbol{\nabla\times}(\boldsymbol{\nabla}f) &= 0\\
    \boldsymbol{\nabla\cdot}(\boldsymbol{\nabla\times F}) &= 0
\end{align}

\subsection{Gradient}
El gradient compleix les relacions següents:
\begin{align}
    \boldsymbol{\nabla}(f+g) &= \boldsymbol{\nabla}f + \boldsymbol{\nabla}g\\
    \boldsymbol{\nabla}(fg) &= f \boldsymbol{\nabla}g + g \boldsymbol{\nabla} f\\
    \boldsymbol{\nabla}(\boldsymbol{F\cdot G}) &=
    \boldsymbol{F\times}(\boldsymbol{\nabla\times G}) + \boldsymbol{G\times}(\boldsymbol{\nabla\times F}) +
    (\boldsymbol{F\cdot\nabla})\boldsymbol{G} + (\boldsymbol{G\cdot\nabla})\boldsymbol{F}
\end{align}

\subsection{Divergència}
La divergència compleix les relacions següents:
\begin{align}
    \boldsymbol{\nabla\cdot}(\boldsymbol{A}+\boldsymbol{B}) &= (\boldsymbol{\nabla\cdot A}) +
    (\boldsymbol{\nabla\cdot B})\\
    \boldsymbol{\nabla\cdot}(f\boldsymbol{F}) &=
    (\boldsymbol{\nabla}f)\boldsymbol{\cdot F} + f(\boldsymbol{\nabla\cdot F})\\
       \boldsymbol{\nabla\cdot}(\boldsymbol{F\times G}) &=
    \boldsymbol{G\cdot}(\boldsymbol{\nabla\times F}) -
    \boldsymbol{F\cdot}(\boldsymbol{\nabla\times G})
\end{align}

\subsection{Rotacional}
El rotacional compleix les relacions següents:
\begin{align}
    \boldsymbol{\nabla\times}(\boldsymbol{A}+\boldsymbol{B}) &= (\boldsymbol{\nabla\times A}) + (\boldsymbol{\nabla\times B})\\
    \boldsymbol{\nabla\times}(f\boldsymbol{F}) &=
    (\boldsymbol{\nabla}f)\boldsymbol{\times F} + f(\boldsymbol{\nabla\times F})\\
    \boldsymbol{\nabla\times}(\boldsymbol{F\times G}) &= (\boldsymbol{G\cdot\nabla})\boldsymbol{F} - (\boldsymbol{F\cdot\nabla})\boldsymbol{G} + \boldsymbol{F}(\boldsymbol{\nabla\cdot G}) - \boldsymbol{G}(\boldsymbol{\nabla\cdot F})
\end{align}


\section{Teoremes Vectorials}

La funció escalar $f$ i la funció vectorial $\boldsymbol{F}$ que apareixen en els teoremes següents han de ser contínues, diferenciables i les seves derivades primeres han de ser funcions contínues.

\subsection{Teorema fonamental del gradient}
Estableix que la integral sobre una corba $C$ de la funció $\boldsymbol{\nabla} f$ només depèn dels punts inicial i final sobre la corba, $P$ i $Q$, i no del camí seguit per anar de l'un a l'altre.
\begin{equation}
    \int_{C_{P\rightarrow Q}} (\boldsymbol{\nabla}f) \cdot \diff l = f(Q)-f(P)
\end{equation}

Evidentment, la integral anterior és nuŀla quan la corba d'integració és tancada ($P \equiv Q$).
\begin{equation}
    \oint_C (\boldsymbol{\nabla}f) \cdot \diff l = 0
\end{equation}

\subsection{Teorema fonamental de la divergència (també anomenat de Gauss o de Green)}
Estableix la relació entre la integral sobre un volum $V$ de la funció $\boldsymbol{\nabla\cdot F}$ i la integral sobre una superfície $S$ de la funció $\boldsymbol{F}$, on $S$ és la superfície tancada que limita $V$.
\begin{equation}
    \iiint_V(\boldsymbol{\nabla\cdot F})\,\diff \tau = \oiint_S\boldsymbol{F\cdot\diff a}
\end{equation}

El vector $\boldsymbol{\diff a}$ apunta sempre cap a fora del volum $V$.

\subsection{Teorema fonamental del rotacional (també anomenat de Stokes)}
Estableix la relació entre la integral sobre una superfície $S$ de la funció $\boldsymbol{\nabla\times F}$ i la integral sobre una corba $C$ de la funció $\boldsymbol{F}$, on $C$ és la corba tancada que limita $S$.
\begin{equation}
    \iint_S(\boldsymbol{\nabla\times F})\boldsymbol{\,\cdot\diff a} =
    \oint_C\boldsymbol{F\cdot\diff l}
\end{equation}

D'aquesta relació es desprèn que la integral de $\boldsymbol{\nabla\times F}$ sobre qualsevol superfície $S$ limitada per la mateixa corba $C$, té el mateix valor.

Evidentment, la integral anterior és nuŀla quan la superfície d'integració és tancada.
\begin{equation}
    \oiint_S(\boldsymbol{\nabla\times F})\boldsymbol{\,\cdot\diff a} = 0
\end{equation}

En la Figura \vref{pic:signe-teo-stockes} s'iŀlustra el conveni de
signes dels vectors $\boldsymbol{\diff l}$ i $\boldsymbol{\diff a}$.

\begin{center}
	\input{Imatges/Stokes.pdf_tex}
	\captionof{figure}{Conveni de signes del teorema de Stokes}
	\label{pic:signe-teo-stockes}
\end{center}


\subsection{Teorema del gradient}
Estableix la relació entre la integral sobre un volum $V$ de la funció $\boldsymbol{\nabla}f$ i la integral sobre una superfície $S$ de la funció $f$, on $S$ és la superfície tancada que limita $V$.
\begin{equation}
    \iiint_V(\boldsymbol{\nabla}f)\diff \tau = \oiint_S f\boldsymbol{\diff a}
\end{equation}

El vector $\boldsymbol{\diff a}$ apunta sempre cap a fora del volum $V$.

\subsection{Teorema del rotacional}
Estableix la relació entre la integral sobre un volum $V$ de la funció $\boldsymbol{\nabla\times F}$ i la integral sobre una superfície $S$ de la funció $\boldsymbol{F}$, on $S$ és la superfície tancada que limita $V$.
\begin{equation}
    \iiint_V(\boldsymbol{\nabla\times F})\,\diff \tau=-\oiint_S
    \boldsymbol{F\times\diff a}
\end{equation}

El vector $\boldsymbol{\diff a}$ apunta sempre cap a fora del volum $V$.

\section{Conversió de coordenades amb la calculadora \textsf{HP Prime}}

Es donen a continuació una sèrie de programes per a la calculadora \textsf{HP Prime}, destinats a realitzar conversions entre els  sistemes de coordenades rectangulars, cilíndriques i esfèriques.

Utilitzarem les funcions de la calculadora \texttt{rectangular\_coordinates} i \texttt{polar\_coordinates}, pensades per fer conversions entre coordenades cartesianes i polars en dues dimensions.

La funció \texttt{Rec\_a\_Cil} pren com a paràmetre les coordenades cartesianes d'un punt en la forma $[x,y,z]$, i torna les coordenades cilíndriques corresponents en la forma $[\rho,\varphi,z]$.
\lstinputlisting[caption={Funció Rec\_a\_Cil},linerange={3-7}]{HPPrime/Calc_Vect.hpprgm}


La funció \texttt{Cil\_a\_Rec} pren com a paràmetre les coordenades cilíndriques d'un punt en la forma $[\rho,\varphi,z]$, i torna les coordenades cartesianes corresponents en la forma  $[x,y,z]$.
\lstinputlisting[caption={Funció Cil\_a\_Rec},linerange={9-13}]{HPPrime/Calc_Vect.hpprgm}

La funció \texttt{Esf\_a\_Cil} pren com a paràmetre les coordenades esfèriques d'un punt en la forma $[r, \theta,\varphi]$, i torna les coordenades cilíndriques corresponents en la forma  $[\rho,\varphi,z]$.
\lstinputlisting[caption={Funció Esf\_a\_Cil},linerange={15-19}]{HPPrime/Calc_Vect.hpprgm}

La funció \texttt{Cil\_a\_Esf} pren com a paràmetre les coordenades cilíndriques  d'un punt en la forma $[\rho,\varphi,z]$, i torna les coordenades esfèriques corresponents en la forma $[r,\theta,\varphi]$.
\lstinputlisting[caption={Funció Cil\_a\_Esf},linerange={21-25}]{HPPrime/Calc_Vect.hpprgm}

La funció \texttt{Rec\_a\_Esf} pren com a paràmetre les coordenades rectangulars  d'un punt en la forma $[x,y,z]$, i torna les coordenades esfèriques corresponents en la forma $[r,\theta,\varphi]$.
\lstinputlisting[caption={Funció Rec\_a\_Esf},linerange={27-30}]{HPPrime/Calc_Vect.hpprgm}

La funció \texttt{Esf\_a\_Rec} pren com a paràmetre les coordenades esfèriques d'un punt en la forma $[r,\theta,\varphi]$, i torna les coordenades rectangulars corresponents en la forma $[x,y,z]$.
\lstinputlisting[caption={Funció Esf\_a\_Rec},linerange={32-35}]{HPPrime/Calc_Vect.hpprgm}


Totes aquestes funcions treballen en el mode angular que estigui actiu en la calculadora en el moment de ser executades.



\section{Bibliografia}

\textsc{H. M. Schey}. \textsl{Div, grad curl and all that, an informal text on vector calculus}.  W. W. Norton and Company, 3rd edition, 1997.

\textsc{Paul Lorrain, Dale R. Corson}. \textsl{Electromagnetic fields and waves}.  W. H. Freeman and Company, 1972.

\textsc{David J. Griffiths, Reed College}. \textsl{Introduction to electrodynamics}. Prentice-Hall, 3rd edition, 1999.

\textsc{Daniel Fleisch}. \textsl{A student's guide to Maxwell's equations}. Cambridge University Press, 2008.


\end{document}
